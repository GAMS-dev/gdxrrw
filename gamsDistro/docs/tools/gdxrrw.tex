\documentclass{article}
\usepackage{amsmath,graphicx,xspace,hyperref}

\newcommand{\secref}[1]{Section~\ref{sec:#1}}
\newcommand{\gdxrrw}{GDXRRW\xspace}
\newcommand{\R}{R\xspace}
\newcommand{\gams}{GAMS\xspace}
\newcommand{\gdx}{GDX\xspace}

\begin{document}

\title{\gdxrrw: Interfacing \gams and \R}
  \author{Michael C. Ferris\thanks{ Computer
    Sciences Department, University of Wisconsin -- Madison, 1210 West
    Dayton Street, Madison, Wisconsin 53706 ({\tt
    ferris@cs.wisc.edu})}
    \space \space Steven Dirkse\thanks { GAMS Development Corp.,
    1217 Potomac Street, NW,  Washington, DC  20007 ({\tt
    sdirkse@gams.com})}  }
\maketitle

\gdxrrw is a suite of utilities to import/export data between \gams
and \R (both of which the user is assumed to have already) and to
call \gams conveniently from \R.  The software
gives \R users the ability to use all the optimization capabilities of
\gams, and allows visualization and other operations on \gams data
directly within \R.

The \gdxrrw tool is unique among the \gdx interface utilities in that
it is an \R extension made available as an \R package.  As such, it is
run as part of an R session or script, not as part of a GAMS run, and it
follows the usual \R package conventions.
\begin{enumerate}
  \item {\bf Installation}: The \R software is designed to be easily
    extended.  Thousands of extension \emph{packages} are freely and
    conveniently available online and can be installed easily using a simple, standard
    procedure.  The \gdxrrw package is one of these.  The latest
    version is available from the \gdxrrw Wiki
    \url{http://support.gams.com/doku.php?id=gdxrrw:interfacing_gams_and_r} in both source and binary
    form, along with a FAQ list,
    some hints and tips on common problems and solutions, and other
    helpful content.  For convenience, the source and binary packages
    are also available in the {\tt gdxrrw} directory of the GAMS distribution.
  \item {\bf Documentation}: The \gdxrrw documentation is installed as
    part of the \gdxrrw package and is available from within \R in the
    usual way.  See the \gdxrrw Wiki for details and hints.
  \item {\bf Examples, tests, and data}: Like many \R packages,
    \gdxrrw comes with examples and data that help a user get started
    using the package.  The \R help system and the \gdxrrw Wiki
    provide pointers to these.  The Wiki also provides hints on
    finding and running the thousands of lines of tests that come with
    the \gdxrrw package.
\end{enumerate}

\end{document}
